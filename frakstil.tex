\startxmlsetups xml:teisetups
        \xmlsetsetup{#1}{*}{-} 
        \xmlsetsetup{#1}{TEI|teiHeader|fileDesc|titleStmt|sourceDesc|listObject|objectName|profileDesc|creation|idno|text|body|front|list|item|div|head|p|choice|abbr|expan|seg|bibl|ref|note|hi}{xml:*}
        \xmlsetsetup{#1}{title[@level='a']}{xml:title_a}
		\xmlsetsetup{#1}{title[@level='s']}{xml:title_s}
       	\xmlsetsetup{#1}{name[@role='Sprecher']}{xml:sprecher}
       	\xmlsetsetup{#1}{name[@role='Erwaehnung']}{xml:erwaehnt}     
       	\xmlsetsetup{#1}{name[@type='Organisation']}{xml:organisation}
       	\xmlsetsetup{#1}{name[@type='Ort']}{xml:place}
\stopxmlsetups

\xmlregistersetup{xml:teisetups}

% Definiere die Grösse der Buchseite und nenne sie _kgparlbuch_
% Die simple Alternative wäre schlicht \setuppapersize[A4]

\definepapersize[kgparlbuch][width=170mm,height=243mm]
\setuppapersize[kgparlbuch]


% Einbindung von StempelGaramondLTPro als mystempel. Beware: Die Schrift sollte mit genau dem $filename im Font-Verzeichnis des TeX-Baums liegen: ../tex/texmf-fonts/fonts/..
\starttypescript[mystempel]
% \definefontsynonym[Human readable]       		[file:filename without extension]
  \definefontsynonym[mystempel-Regular]    	[file:StempelGaramondLTPro-Roman]
  \definefontsynonym[mystempel-Italic]     		[file:StempelGaramondLTPro-Italic]
  \definefontsynonym[mystempel-Bold]       	[file:StempelGaramondLTPro-Bold]
  \definefontsynonym[mystempel-BoldItalic] 		[file:StempelGaramondLTPro-BoldIt]
\stoptypescript

\starttypescript[mystempel]
  \setups[font:fallback:serif]          % security: if not found==> back to defaults
% \definefontsynonym[ConTeXt basics name] 	[Human readable]      	             [features=default]
  \definefontsynonym[Serif]                [mystempel-Regular]   		 [features=default]
  \definefontsynonym[SerifItalic]         	[mystempel-Italic]    			 [features=default]
  \definefontsynonym[SerifBold]           	[mystempel-Bold]      	 	 [features=default]
  \definefontsynonym[SerifBoldItalic]      	[mystempel-BoldItalic]		 	 [features=default]
\stoptypescript

\starttypescript[mystempel]
  \definetypeface[mystempel]    [rm] [serif] [mystempel]    [default]
\stoptypescript


\mainlanguage[de]
\setupbodyfont[mystempel]
\setupbodyfont[10pt]
\setupwhitespace[small]
\setuptolerance[tolerant]
\setuppagenumbering
			[location={footer,center}]
			
% Kleiner Lua-Schnipsel, um den aktuellen Dateinamen in den Header zu packen			
\setupheadertexts[Quelldatei = \ctxlua{context(environment.inputfilename)}]
		[PDF erzeugt am \currentdate]




\startxmlsetups xml:TEI
	\xmlflush{#1}
\stopxmlsetups

\startxmlsetups xml:teiHeader
	\xmlflush{#1}
\stopxmlsetups

\startxmlsetups xml:fileDesc
	\xmlflush{#1}
\stopxmlsetups

\startxmlsetups xml:titleStmt
	\xmlflush{#1}
\stopxmlsetups

\startxmlsetups xml:title_a
	\title{Irgendwas kann hier rein »\xmlflush{#1}« packen!}
\stopxmlsetups

\startxmlsetups xml:profileDesc
	\xmlflush{#1}
\stopxmlsetups

\startxmlsetups xml:creation
	\xmlflush{#1}
\stopxmlsetups

\startxmlsetups xml:idno
	\xmlflush{#1}
\stopxmlsetups

\startxmlsetups xml:text
	\xmlflush{#1}
\stopxmlsetups

\startxmlsetups xml:front
	\xmlflush{#1}
\stopxmlsetups

\startxmlsetups xml:list
	\xmlflush{#1}
\stopxmlsetups

\startxmlsetups xml:item
	\startitemize
	\item\xmlflush{#1}
	\stopitemize
\stopxmlsetups

\startxmlsetups xml:body
    \xmlflush{#1}
\stopxmlsetups


\startxmlsetups xml:div
	\blank[line]
	\xmlflush{#1}
\stopxmlsetups

\startxmlsetups xml:head
	{\bf \xmlflush{#1}}
	\blank[small]
\stopxmlsetups

\startxmlsetups xml:p
		\xmldoifnotselfempty {#1} {
        \dontleavehmode
        \ignorespaces
        \xmlflush{#1}
        \removeunwantedspaces
    }
    \par	
\stopxmlsetups

\startxmlsetups xml:sprecher
	{\bf
	\xmlflush{#1}}
\stopxmlsetups

\startxmlsetups xml:erwaehnt
	{\it
	\xmlflush{#1}}
\stopxmlsetups

\startxmlsetups xml:organisation
	\xmlflush{#1}
\stopxmlsetups

\startxmlsetups xml:place
	\xmlflush{#1}
\stopxmlsetups

\startxmlsetups xml:choice
	\xmlflush{#1}
\stopxmlsetups

\startxmlsetups xml:abbr
	\xmlflush{#1}
\stopxmlsetups

\startxmlsetups xml:expan
	\startfootnote
		\xmlflush{#1}
	\stopfootnote
\stopxmlsetups

\startxmlsetups xml:seg
		\xmlflush{#1}
\stopxmlsetups


\startxmlsetups xml:note
	\startfootnote
		\xmlflush{#1}
	\stopfootnote	
\stopxmlsetups

\startxmlsetups xml:bibl
		\xmlflush{#1}
\stopxmlsetups

\startxmlsetups xml:ref
		\xmlflush{#1}
\stopxmlsetups

\startxmlsetups xml:hi
		{\smallcaps
		\xmlflush{#1}}
\stopxmlsetups
